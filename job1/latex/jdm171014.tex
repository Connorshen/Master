\documentclass[twocolumn]{article}
\usepackage[utf8]{inputenc}
\usepackage{ctex}
\usepackage[T1]{fontenc}
\usepackage[english]{babel}
\usepackage{ifpdf,amsmath,amsthm,amssymb,amsfonts,newtxtext,newtxmath} 
\usepackage{array,graphicx,dcolumn,multirow,hevea,abstract,hanging}
\usepackage[labelfont=sc,textfont=sf]{caption}
\usepackage[hyperfootnotes=false,breaklinks=true]{hyperref} % was dvipdfmx
\urlstyle{rm}
%\usepackage[hyphenbreaks]{breakurl}
\usepackage{apacite} % must come afer hyperfootnotes
\setlength{\bibleftmargin}{1em}
\setlength{\bibindent}{-1em}
\usepackage{booktabs} % \toprule \midrule \bottomrule \cmidrule(lr){a-b}

%%%% Added by Zhenbo Cheng
\usepackage{subcaption}
%%%%%%%%%%%%%%%%%%%%%%%%

% define centered and ragged columns:
\newcolumntype{L}[1]{>{\raggedright\arraybackslash }p{#1}} % can use m{}
\newcolumntype{C}[1]{>{\centering\arraybackslash }p{#1}}
\newcolumntype{R}[1]{>{\raggedleft\arraybackslash }p{#1}}
\newcolumntype{d}[1]{D{.}{.}{#1}} % d{3.2} for 3 places on l, 2 on r
\newcommand{\mc}{\multicolumn}
\topmargin=-.3in \oddsidemargin=-.1in \evensidemargin=-.1in \textheight=9in \textwidth=6.8in
\setlength\tabcolsep{1mm}
\setlength\columnsep{5mm}
\setlength\abovecaptionskip{.5ex}
\setlength\belowcaptionskip{.5ex}
\setlength\belowbottomsep{.3ex}
\setlength\lightrulewidth{.04em}
\renewcommand\arraystretch{1.2}
\renewcommand{\topfraction}{1}
\renewcommand{\textfraction}{0}
\renewcommand{\floatpagefraction}{.9}
% \renewcommand{\baselinestretch}{1.00} \large\normalsize % for fixing spaces
\widowpenalty=1000
\clubpenalty=1000
\setlength{\parskip}{0ex}
\let\tempone\itemize
\let\temptwo\enditemize
\let\tempthree\enumerate
\let\tempfour\endenumerate
\renewenvironment{itemize}{\tempone\setlength{\itemsep}{0pt}}{\temptwo}
\renewenvironment{enumerate}{\tempthree\setlength{\itemsep}{0pt}}{\tempfour}

%%%%%%%%%%%%%%%%%%%%%%%%%%%%%%%%%%%%%%%%%%%%%%%%%%%%%%%%%%%%%%%%%%%%%
\setcounter{page}{1} % start with first page

\title{Water Quality Modeling and Prediction Method Based on Sparse Recurrent Neural Network}

\author{
Zhenbo Cheng
}

% who is corresponding author?

\date{} % leave empty
\begin{document} % goes here

% fill in short title
\newcommand{\jref}{http://journal.sjdm.org/vol13.2.html}
\newcommand{\jhead}{Judgment and Decision Making, Vol.~13, No.~2}
\newcommand{\jdate}{March 2018}
\pagestyle{myheadings} \markright{\protect\small \href{\jref}{\jhead}, \jdate \hfill Strategies of matcching and maxmizing \qquad}
\begin{htmlonly}
\href{\jref}{\jhead}, \jdate, pp.\
\end{htmlonly}
%\begin{latexonly}
\twocolumn[
\vspace{-.3in}
{\small \href{\jref}{\jhead}, \jdate, pp.\ XX--XX}
%\end{latexonly}

\maketitle

%\begin{latexonly}
\vspace{-3mm}
\begin{onecolabstract}
%\end{latexonly}
It is an important prerequisite for scientific management and maintenance of water 
resources to accurately predict all kinds of indicators that affect water quality. 
This paper proposed a method of forecasting water quality index and rank based on sparse 
recurrent neural network (SRNN). Based on the principle of minimum mean square 
recursive error, the training algorithm of the network was designed. The neural
 network was used to construct a water quality prediction model. 
 The experimental results showed that the model can be used to predict the 
 trend of water quality in ZheJiang province.
\smallskip

\noindent
Keywords: water quality modeling,water quality prediction,sparse recurrent neural network,learning algorithm
%\begin{latexonly}
\end{onecolabstract}\bigskip
]
%\end{latexonly}

{\renewcommand{\thefootnote}{}
\footnotetext{ % note blank lines above and below acknowledgment

  This work is supported by Public Projects of Zhejiang Province
  (2016C31G2020069) and the 3rd Level in Zhejiang Province “151
  talents project” to Zhenbo Cheng. We thank Liwen Bianji, Edanz Group
  China (www.liwenbianji.cn/ac), for editing the English text of a
  draft of this manuscript. The authors declare no competing financial
  or nonfinancial interests.

Copyright: \copyright\ 2018.
The authors license this article under the terms of the
\href{http://creativecommons.org/licenses/by/3.0/}{Creative Commons
  Attribution 3.0 License.}
}}

\saythanks

\setlength{\baselineskip}{12pt plus.2pt}

\section{Introduction}
abc\cite{RN1}

\bibliographystyle{apacite}
\bibliography{mybib.bib}



\bigskip


\end{document}

